\begin{figure}[h]
    
    \begin{tikzpicture}[scale=.95]  
        % Ports
        \fill[red!5] ($(0, 0) + (30:6)$) arc (30:150:6) -- ($(0, 0) + (150:7)$) arc (150:30:7) -- cycle;
        \fill[red!5] ($(0, 0) + (150:6)$) arc (150:270:6) -- ($(0, 0) + (270:7)$) arc (270:150:7) -- cycle;
        \fill[red!5] ($(0, 0) + (-90:6)$) arc (-90:30:6) -- ($(0, 0) + (30:7)$) arc (30:-90:7) -- cycle;

        % Adapaters
        \fill[yellow!5] ($(0, 0) + (30:5)$) arc (30:150:5) -- ($(0, 0) + (150:6)$) arc (150:30:6) -- cycle;
        \fill[yellow!5] ($(0, 0) + (150:5)$) arc (150:270:5) -- ($(0, 0) + (270:6)$) arc (270:150:6) -- cycle;
        \fill[yellow!5] ($(0, 0) + (-90:5)$) arc (-90:30:5) -- ($(0, 0) + (30:6)$) arc (30:-90:6) -- cycle;

        % Controllers
        \fill[gray!5] ($(0, 0) + (30:4)$) arc (30:150:4) -- ($(0, 0) + (150:5)$) arc (150:30:5) -- cycle;
        \fill[gray!5] ($(0, 0) + (150:4)$) arc (150:270:4) -- ($(0, 0) + (270:5)$) arc (270:150:5) -- cycle;
        \fill[gray!5] ($(0, 0) + (-90:4)$) arc (-90:30:4) -- ($(0, 0) + (30:5)$) arc (30:-90:5) -- cycle;

        % Dispatcher
        \fill[blue!5] (0,0)circle(4);

        % UseCases
        \fill[green!5] ($(0, 0) + (30:2)$) arc (30:150:2) -- ($(0, 0) + (150:3)$) arc (150:30:3) -- cycle;
        \fill[green!5] ($(0, 0) + (150:2)$) arc (150:270:2) -- ($(0, 0) + (270:3)$) arc (270:150:3) -- cycle;
        \fill[green!5] ($(0, 0) + (-90:2)$) arc (-90:30:2) -- ($(0, 0) + (30:3)$) arc (30:-90:3) -- cycle;

        % Interactors
        \fill[olive!5] (0,0)circle(2);

        \path[font=\footnotesize]

        (0, -0.2) node[above]{$Domain$}
        (0, 1.2) node[above]{$Interactors$}
        (0, 2.2) node[above]{$Use Cases$}
        (0, 3.2) node[above]{$Dispatcher$}
        (0, 4.2) node[above]{$Controllers$}
        (0, 5.2) node[above]{$Adapters$}
        (0, 6.2) node[above]{$Ports$};

        % Dataflow
        \coordinate[label=below:$A$] (nodeA) at ($(0,0) + (225:7.5)$);
        \coordinate[label=below:$B$] (nodeB) at ($(0,0) + (225:6.5)$);
        \coordinate[label=below:$C$] (nodeC) at ($(0,0) + (225:5.5)$);

        \coordinate[label=below:$D$] (nodeD) at ($(0,0) + (225:4.5)$);
        \coordinate[label=below:$E$] (nodeE) at ($(0,0) + (225:3.5)$);
        \coordinate[label=above:$G$] (nodeG) at ($(0,0) + (225:2.5)$);
        \coordinate[label=above:$K$] (nodeK) at ($(0,0) + (225:1.5)$);
        \coordinate[label=right:$L$] (nodeL) at ($(0,0) + (225:-4.5)$);

        \coordinate[label=below:$M$] (nodeM) at ($(0,0) + (225:-5.5)$);
        \coordinate[label=below:$N$] (nodeN) at ($(0,0) + (225:-6.5)$);
        \coordinate[label=below:$O$] (nodeO) at ($(0,0) + (225:-7.5)$);

        \draw[teal] (nodeA) -- (nodeB);
        \draw[teal] (nodeB) -- (nodeC);
        \draw[teal, ->] (nodeC) -- (nodeD); 

        \draw[blue] (nodeD) -- (nodeE); 
        \draw[blue] (nodeE) -- (nodeG); 
        \draw[blue] (nodeG) -- (nodeK);
        \draw[blue, ->, dotted](nodeK) -- (nodeL);

        \draw[red](nodeL) -- (nodeM);
        \draw[red] (nodeM) -- (nodeN);
        \draw[red, ->] (nodeN) -- (nodeO);

        % Legends
        \draw (0, -8) -- (1, -8) node[right] {$Teile\ des\ Programms\ wissen\ nichts\ voneinander$};
    \end{tikzpicture}
    \caption[some Caption]{Darstellung der umgesetzten Architektur als \textbf{Clean Architecture} mit sieben Schichten. 
    Pfeile repräsentieren mögliche Datenflüsse \footnotemark}
\end{figure}