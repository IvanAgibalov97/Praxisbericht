\begin{figure}[h]
    
    \begin{tikzpicture}[scale=.95]  
        % Ports
        \fill[red!5] ($(0, 0) + (30:6)$) arc (30:150:6) -- ($(0, 0) + (150:7)$) arc (150:30:7) -- cycle;
        \fill[red!5] ($(0, 0) + (150:6)$) arc (150:270:6) -- ($(0, 0) + (270:7)$) arc (270:150:7) -- cycle;
        \fill[red!5] ($(0, 0) + (-90:6)$) arc (-90:30:6) -- ($(0, 0) + (30:7)$) arc (30:-90:7) -- cycle;

        % Adapaters
        \fill[yellow!5] ($(0, 0) + (30:5)$) arc (30:150:5) -- ($(0, 0) + (150:6)$) arc (150:30:6) -- cycle;
        \fill[yellow!5] ($(0, 0) + (150:5)$) arc (150:270:5) -- ($(0, 0) + (270:6)$) arc (270:150:6) -- cycle;
        \fill[yellow!5] ($(0, 0) + (-90:5)$) arc (-90:30:5) -- ($(0, 0) + (30:6)$) arc (30:-90:6) -- cycle;

        % Controllers
        \fill[gray!5] ($(0, 0) + (30:4)$) arc (30:150:4) -- ($(0, 0) + (150:5)$) arc (150:30:5) -- cycle;
        \fill[gray!5] ($(0, 0) + (150:4)$) arc (150:270:4) -- ($(0, 0) + (270:5)$) arc (270:150:5) -- cycle;
        \fill[gray!5] ($(0, 0) + (-90:4)$) arc (-90:30:4) -- ($(0, 0) + (30:5)$) arc (30:-90:5) -- cycle;



        % Dispatcher
        \fill[blue!5] (0,0)circle(4);

        % UseCases
        \fill[green!5] ($(0, 0) + (30:2)$) arc (30:150:2) -- ($(0, 0) + (150:3)$) arc (150:30:3) -- cycle;
        \fill[green!5] ($(0, 0) + (150:2)$) arc (150:270:2) -- ($(0, 0) + (270:3)$) arc (270:150:3) -- cycle;
        \fill[green!5] ($(0, 0) + (-90:2)$) arc (-90:30:2) -- ($(0, 0) + (30:3)$) arc (30:-90:3) -- cycle;

        % Interactors
        \fill[olive!5] ($(0, 0) + (30:1)$) arc (30:150:1) -- ($(0, 0) + (150:2)$) arc (150:30:2) -- cycle;
        \fill[olive!5] ($(0, 0) + (150:1)$) arc (150:270:1) -- ($(0, 0) + (270:2)$) arc (270:150:2) -- cycle;
        \fill[olive!5] ($(0, 0) + (-90:1)$) arc (-90:30:1) -- ($(0, 0) + (30:2)$) arc (30:-90:2) -- cycle;

        % Domain
        \fill[cyan!5] (0,0)circle(1);
        %Add labels with names of the primary and secondary colors.
        
        % Boundaries between components
        \draw ($(0,0) + (30:4)$) -- (30:7);
        \draw ($(0,0) + (150:4)$) -- (150:7);
        \draw ($(0,0) + (270:4)$) -- (270:7);

        \draw[dotted] (0,0)circle(1);
        \draw[dotted] (0,0)circle(2);
        \draw[line width = 1pt, dotted] (0,0)circle(3);
        \draw[line width = 1pt, dotted] (0,0)circle(4);
        \draw[dotted] (0,0)circle(5);
        \draw[dotted] (0,0)circle(6);
        \draw (0,0)circle(7);

        % Boundaries between Usecases
        \foreach \x in {0, 30, ..., 330}
            \draw ($(0,0) + (\x:2)$) -- (\x:3);


        \path[font=\footnotesize]

        (0, -0.2) node[above]{$Domain$}
        (0, 1.2) node[above]{$Interactors$}
        (0, 2.2) node[above]{$Use Cases$}
        (0, 3.2) node[above]{$Dispatcher$}
        (0, 4.2) node[above]{$Controllers$}
        (0, 5.2) node[above]{$Adapters$}
        (0, 6.2) node[above]{$Ports$};

        % Legends
        \draw (0, -8) -- (1, -8) node[right] {$Teile\ des\ Programms\ wissen\ nichts\ voneinander$};
    \end{tikzpicture}
    \caption[some Caption]{Darstellung der umgesetzten Architektur als \textbf{Clean Architecture} mit 7 Schichten. 
    Die Linien repräsentieren die Grenzen zwischen den einzelnen Teilen des Programms \footnotemark}
    \label{fig:sp2d}
\end{figure}