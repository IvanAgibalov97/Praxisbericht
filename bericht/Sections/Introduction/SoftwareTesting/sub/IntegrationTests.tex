
Die Integration Tests stellen fest, ob die zusammengesetzte Module, Komponenten oder Klassen, welche die Unit Tests bestanden haben, wie gewünscht funktionieren.

Alle anderen verwendeten Module werden analog zu den Unit Tests manipuliert.

Die Anzahl an Integration Tests in einem Projekt ist geringer als Anzahl an Unit Tests.
Auf der einen Seite ist ein Integration Tests größer als ein Unit Test. 
Andererseits ist die Fehlersuche im Programm bei negativem Testergebnis deutlich komplexer und zeitintensiver, 
als bei Unit Tests.

Der Vorteil von Integrationstests liegt darin, dass mit ihrer Hilfe ein großer Teil des Codes mit Tests abgedeckt werden kann.