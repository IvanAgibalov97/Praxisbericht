Unit Tests repräsentieren die Mehrzahl an Tests in einem Projekt. Schlägt ein Unit Test fehl, 
kann der Fehler im Programm sofort lokalisiert und dadurch die Dauer der Fehlersuche minimiert werden.
Da sie nur einen kleinen Teil des Programms abdecken, 
sind sie am meistens in einem Projekt repräsentiert und haben die geringste Laufzeit im Vergleich zu den anderen Tests.
Unit Tests überprüfen, ob die kleinsten Module (meistens einzelne Funktionen und Objekte), 
wie gewünscht funktionieren. 
Alle Module, welche das zu testende Objekt verwendet, werden während der Ausführung von Unit Tests manipuliert.
Auf diese Weise können diverse Situationen simuliert werden.