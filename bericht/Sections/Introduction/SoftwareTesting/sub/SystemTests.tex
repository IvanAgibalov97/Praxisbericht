Bei Systemtests wird die zu prüfende Software wie ein Produktivsystem
% Bei den Systemtests wird das gleiche System wie in \textbf{production} 
gestartet und entsprechend getestet. 
Alle externen Schnittstellen werden im Test simuliert und jeder Test wird nur an Ausgaben an externen Schnittstellen validiert.
Da die komplette Anwendung getestet wird, ist die Anzahl an benötigten Tests geringer als bei Integrationstests. 
Die Laufzeit jedes einzelnen Tests ist sehr groß 
und die Fehlersuche dauert länger, da der Fehler sich schwer in der Anwendung lokalisieren lässt.