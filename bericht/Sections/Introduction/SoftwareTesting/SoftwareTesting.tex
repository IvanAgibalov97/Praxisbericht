Jede Software wird im Laufe der Zeit viel geändert und erweitert.
Die erste Version einer Software kann beispielsweise mit einer Version der selben Software nach 10 Jahren, wenig bis keine Gemeinsamkeiten besitzen.
Jede Änderung des Quellcodes ist ein Risiko für Softwareentwickler, da immer die Gefahr besteht versehentlich funktionierenden Code zu beschädigen.

Um das Risiko für Defekte zu minimieren, können automasierte Tests verwendet werden. Je nach Ausführung stellen diese fest,
ob das bestehende Verhalten noch dem Sollverhalten entspricht. Hierbei wird nur das Verhalten geprüft, 
welches mit den entsprechenden Tests abgedeckt wurde.

Es gibt mehrere Typen von automasierten Tests, die im Folgenden beschrieben werden.

