

Jede Software erlebt im Laufe der Zeit sehr viele Änderungen. 
Die erste Version von Software kann nichts gemeinsames mit der Software nach 10 Jahren haben.
Jede Änderung des Quellcodes ist ein Risiko für den Softwareentwickler(in), da man immer irgendwas ausversehen kaputt machen kann.

Damit man das Risiko die Software kaputt zu machen minimiert, kann man dies mit Automatisieren Tests abdecken, die nach jeder Ausführung festellen, ob das bestehende Verhalten (nur das Verhalten, 
das entsprechend mit Tests abgedect wurde) noch genau so ist.

Es gibt mehrere Typen von automasierten Tests, die hier kurz beschrieben werden.

