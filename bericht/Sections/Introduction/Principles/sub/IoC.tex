Inversion of control (IoC, dt. Steuerungsumkehr) ist ein Prinzip der Softwareentwicklung, 
bei dem die Ausführung vom geschriebenen Code an andere Stelle ausgelagert wird
(z.B. ein Framework)\cite{IoCDefinition}.
% 

Beispiel für IoC ist die Reaktion auf ein Ereignis mittels einem Callback. 
Es wird eine Methode (Callback) definiert, die an eine externe Stelle übergeben wird und 
beim Geschehen des Ereignisses zum späteren Zeitpunkt ausgeführt wird. 
Die Kontroller darüber, wann das Ereignis geschieht und wie die Zuordnung des Callbacks passiert, geht dadurch verloren. 