Das Interface-Segregation-Prinzip (dt. Schnittstellenaufteilungsprinzip) 
ist ein Begriff aus dem objektorientierten Entwurf. Er besagt dass eine große Schnittstelle (Interface) 
in mehrere kleine Schnittstellen aufgeteilt werden soll.

Beispiel: es gibt eine Schnittstelle, die \textbf{IDatabase} heißt. 
Diese Schnittstelle definiert folgende Methoden:
\begin{itemize}
    \item getUser()
    \item setUser()
    \item getTransaction()
    \item setTransaction()
    \item getCharger()
    \item setCharger()
\end{itemize}

In der definierten Schnittstelle ist zu sehen, 
dass es sich in drei Gruppen aufteilen lässt (jeweils Methoden für Charger, User und Transaction).
\begin{enumerate}
    \item getUser(), setUser()
    \item getTransaction(), setTransaction(),
    \item getCharger(), setCharger()
\end{enumerate}

Nach \textbf{ISP} sollen besser drei statt eine Schnittstellen definiert werden.
Zum Beispiel mit folgenden Namen:
\begin{enumerate}
    \item IDatabaseUser: getUser(), setUser(),
    \item IDatabaseCharger: getCharger(), setCharger(),
    \item IDatabaseTransaction: getTransaction(), setTransaction()
\end{enumerate}

Der Nutzen von \textbf{ISP} ist,
dass wenn mehrere Interfaces gleiche Methode besitzen sollen, 
ist es am besten die gleichen Methoden in ein anderes Interface zu isolieren.
Das bringt ein besseres Verständnis über die Struktur der Anwendung und 
minimiert Anzahl an Teilen der Software, 
die im Falle einer Änderung, mitgeändert werden sollen.