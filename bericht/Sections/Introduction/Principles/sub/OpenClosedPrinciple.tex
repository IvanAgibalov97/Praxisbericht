Das Open-Closed-Prinzip (dt. Prinzip der Offen- und Verschlossenheit) beschäftigt sich 
mit Erweiterbarkeit und Änderbarkeit einer Software.

Die Definition von OCP ist:
\textit{A software artifact should be open for extension but 
closed for modification}\cite[70]{cleanArchitecture}

Wenn eine neue Erweiterungen in der bestehenden Software gemacht werden soll, 
darf nach dem \textbf{OCP} die bestehende Software minimal geändert werden 
(idealerweise gar nicht). Das lässt sich in der Kombination mit \textbf{SRP} (siehe Kapitel \ref{kap:SRP})
erreichen, indem alle Module bzw. Klassen nur eine Aufgabe erledigen und somit 
nur einen Grund für die Änderungen besitzen.
