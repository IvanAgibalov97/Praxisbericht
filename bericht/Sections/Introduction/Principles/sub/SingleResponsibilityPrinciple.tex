Das Single Responsibility Prinzip besagt, dass es nie mehr als einen Grund geben sollte,
ein Modul (z.B. eine Klasse) zu ändern. Zum Beispiel wenn ein Modul zwei Aufgaben erledigt, die sehr wenig 
oder gar nicht miteinander verbunden sind, ist es ein Signal das Modul in zwei Module aufzuteilen.

Es lässt sich auch überprüfen, indem die Aufgabe und Verantwortlichkeit von einem Modul 
in einem Satz zusammengefasst werden. Wenn es dabei das Word ``und'' benutzt wird, ist es ein Zeichnen, dass
das Modul aufgeteilt werden soll.

% sehe SRP Clean Code S.138


% https://www.it-economics.de/blog/2014-06/2014-das-single-responsibility-principle