Mehr zu MVP (Model-View-Presenter) kann man \textit{\textbf{in folgende Quelle}} lesen.

Model-View-Presenter Architektur wird in den Anwendungen benutzt, die eine Oberfläche besitzen.
Die Architektur teilt die Anwendung in drei Teile:
\begin{itemize}
    \item \textbf{Model} - enthält die komplete Logik des Programms.
    \item \textbf{View} - empfängt alle Ereignisse von der Benutzeroberfläche (UI) und enthält die Daten, die angezeigt werden sollen.
    \item \textbf{Presenter} - transformiert die Daten in beide Richtungen vom Model zu View 
    und vom View zu Model
\end{itemize}


Eigenschaften der MVP Architektur:
\begin{itemize}
    \item \textbf{Presenter} und \textbf{Model} lassen sich mit Unittests abdecken.
    \item Jedes neues \textbf{View} braucht ein eigenes \textbf{Presenter}.
\end{itemize}

\begin{figure}[H]
    \centering
    \includegraphics[width=1\textwidth]{./images/MVP.png}
    \caption[Datenfluss in MVP Architektur]{Datenfluss in MVP Architektur}
    \label{fig:MVP}
\end{figure}

Der Datenfluss in einer Anwendung, die mittels MVP implementiert ist:
\begin{itemize}
    \item In der Oberfläche (View) wird ein Ereignis erzeugt(z.B. ein Button wurde gedrückt).
    \item View gibt das Ereignis an Presenter weiter.
    \item Das Ereignis wird im Presenter einem im Model definierten Ereignis zugeordnet(z.B. Button "Speichern" wurde gedrückt)
    \item Das Model bearbeitet das Ereignis (z.B. die Datei wurde gespeichert) und gibt das Ergebnis zurück.
    \item Presenter empfängt das Ergebnis (z.B. die Datei wurde erfolgreich gespeichert) und wandelt das in ein Ereignis,
    das View interpetieren kann (z.B. Button "Speicher" soll grün werden).
    \item View abarbeit das Ergebnis und zeigt es an (z.B. Button wird grün auf der Oberfläche angezeigt)
\end{itemize}