Das Interface-Segregation-Prinzip (dt. Schnittstellenaufteilungsprinzip) 
ist Begriff aus des objektorientierten Entwurfs. Er besagt dass eine große Schnittstelle (Interface) 
in mehrere kleine Schnittstellen aufgeteilt werden soll.

Beispiel: es gibt eine Schnittstelle, die \textbf{IDatebase} heißt. 
Diese Schnittstelle definiert folgende Methoden:
\begin{itemize}
    \item getUser()
    \item setUser()
    \item getTransaction()
    \item setTransaction()
    \item getCharger()
    \item setCharger()
\end{itemize}

In der definierten Schnittstelle ist zu sehen, 
dass es sich in drei Gruppen aufteilen lässt (jeweils Methoden für Charger, User und Transaction).
\begin{enumerate}
    \item getUser(), setUser()
    \item getTransaction(), setTransaction(),
    \item getCharger(), setCharger()
\end{enumerate}

Nach \textbf{ISP} sollen besser drei statt eine Schnittstellen definiert werden.
Zum Beispiel mit folgenden Namen:
\begin{enumerate}
    \item IDatebaseUser: getUser(), setUser(),
    \item IDatebaseCharger: getCharger(), setCharger(),
    \item IDatebaseTransaction: getTransaction(), setTransaction()
\end{enumerate}