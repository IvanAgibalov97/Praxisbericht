Das Dependency inversion principle(dt. Abhängigkeits-Umkehr-Prinzip) oder DIP ist ein Prinzip 
beim objektorientierten Entwurf von Software. Es beschäftigt sich mit der Abhängigkeit von Modulen.

Im Allgemeinen wird das DIP beschrieben durch:

Module höherer Ebenen sollten nicht von Modulen niedrigerer Ebenen abhängen.
Beide sollten von Abstraktionen abhängen.


Abstraktionen sollten nicht von Details abhängen.
Details sollten von Abstraktionen abhängen.

Laut DIP, die Aussage ``Modul A benutzt Modul B'' soll nie direkt passieren, sondern über ein Interface I.

\begin{figure}[H]
    \centering
    \includegraphics[width=1\textwidth]{./images/DIP - bad.png}
    \caption[Schlechte laut DIP Abhängigkeit]{Schlechte laut DIP Abhängigkeit \footnotemark}
    \label{fig:MVP}
\end{figure}


\begin{figure}[H]
    \centering
    \includegraphics[width=1\textwidth]{./images/DIP - good.png}
    \caption[Gute laut DIP Abhängigkeit]{Gute laut DIP Abhängigkeit \footnotemark}
    \label{fig:MVP}
\end{figure}

% https://dewiki.de/Lexikon/Dependency-Inversion-Prinzip