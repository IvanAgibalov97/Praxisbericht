Factory method (dt. Fabrikmethode) ist ein OOP Design Pattern, mit dem die Instanzierung nicht direkt mittels eines Konstruktors der Instanz 
sondern indirekt mittels einer Methode einer anderen Instanz (Fabrik), die die Aufgabe der Instanzierung übernimmt.

Die Vorteile der Übergabe der Instanzierung von Objekten an eine andere Instanz sind:
\begin{itemize}
    \item Instanzierte Objekte lassen sich überwachen (z.B. zählen)
    \item Komplexe Instanzierung (z.B. mehrere Funktionsaufrufe) lassen sich hinter eine Methode verbergen.
    \item Lange Parameterliste lässen sich kürzen, indem die gleichen Parameter werden von der Fabrik übergeben.
\end{itemize}