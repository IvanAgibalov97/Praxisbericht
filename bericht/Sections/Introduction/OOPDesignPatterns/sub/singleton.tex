Mehr zu Singleton kann \textit{\textbf{Refactoring guru}} lesen.

Singleton (dt. Einzelstück) garantiert, dass eine Klasse nur eine Instanz in der Anwendung hat.
Dieser Pattern kann sehr hilfreich sein, wenn man weißt, man hat nur eine Instanz der Klasse im Programm 
und auf sie möchte man aus allen Stellen der Anwendung zugreifen.
Das Verhalten ähnelt sich mit dem Verhalten einer 
globalen Variable deren Wert (die erstellte Instanz) nicht ersetzt werden kann.

Ein großer Nachteil bei der Benutzung des Singletons ist, 
dass die Teile des Programms, in denen Singleton aufgerufen wird, nicht unabhängig von ihm getestet werden können.