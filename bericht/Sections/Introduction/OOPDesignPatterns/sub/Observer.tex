Das OOP Design Pattern \textbf{Observer} ermöglicht dynamische Verbindungen zwischen den einzelnen 
Objecten im Programm, um über die geschehenen Ereignisse im Programm alle Interessenten 
zu informieren.

Das Pattern besteht aus 2 Teilen: \textbf{Publisher} und \textbf{Observers} oder \textbf{Subscribers}

\textbf{Subscribers} können bestimmte Events im \textbf{Publisher} abonnieren und disabonnieren.
\textbf{Publisher} informiert alle auf das geschehene Event abonnierten \textbf{Subscribers}, wenn es auftritt. 
\textbf{Subscribers} können dann gewisses Verhalten auf das Event definieren.


\begin{figure}[H]
    \centering
    \includegraphics[width=1\textwidth]{Images/Observer.png}
    \caption[UML Observer]{Klassendiagrammm Observer}
    \label{fig:flow around cylinder}
    \source{Eigene Quelle}
\end{figure}