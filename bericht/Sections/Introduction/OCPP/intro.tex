Das OCPP (engl. Open Charge Point Protocol) ist ein Kommunikationsstandard zwischen der Ladesäule (Charging Station) und dem Server (Charging Station Management System)
und wurde enwickelt um jegliche Ladetechnik zu unterstützen.\cite[Part 0, 3. Seite]{ocppDocs}

Mithilfe von standardisierter OCPP Kommunikation können die Software für die Ladesäule und den Server unabhängig voneinander geschrieben werden.
Somit kann ein Server Ladesäulen von unterschiedlichen Herstellern gleichzeitig betreiben.

OCPP definiert eine Menge an Nachrichten, deren Inhalte und Struktrur. 
Es werden auch die Situationen beschrieben, in deren die Nachrichten abgeschickt werden dürfen.
Das Übertragungsprotokoll, in dem der Nachrichtenaustausch stattfindet, ist vorgegeben.

Für diese Arbeit sind folgende Eigenschaften des OCPP-Protokolls von Bedeutung:
\begin{itemize}
    \item Übertragungsprotokoll ist WebSocket, dadurch wird die Implementation des \textbf{Ports} vorgegeben (siehe Kapitel \ref{kap:commonArchitectureDescription})
    \item Inhalt und Struktur jeder Nachricht, gibt das Verhalten des \textbf{Adapters} vor (siehe Kapitel \ref{kap:commonArchitectureDescription})
    \item Vordefiniertes Verhalten gibt eine Menge an \textbf{UseCases} vor. (siehe Kapitel \ref{kap:commonArchitectureDescription})
\end{itemize}