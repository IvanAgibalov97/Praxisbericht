Inversion of control (IoC) ist ein Prinzip der Softwareentwicklung, 
bei dem die Ausführung vom geschriebenen Code von einem Externen zu dem 
richtigen Zeitpunkt ausgeführt wird.

Das Prinzip entkoppelt die Teile der Anwendung voneinander, indem die Reaktion nicht mehr vom 
Erstllen gebunden ist.

Beispiel für IoC ist die Reaktion auf ein Ereignis mittels einem Callback. 
Man schreibt eine Methode(Callback), die an eine externe Stelle übergeben wird und 
beim Geschehen des Ereignisses zum späteren Zeitpunkt ausgeführt wird. 
Man hat dabei keine Kontrolle darüber, wann das Ereignis geschieht und wie die Zuordnung des Callbacks passiert.

Ein weiteres Beispiel von IoC ist ``Dependency Injection''