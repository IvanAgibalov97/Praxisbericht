\section{Gewünschtes Interfaces}
    Jedes Tool, Bibliothek, Framework, das von den anderen Menschen benutzt wird, 
    soll soweit wie möglich selbsterklärend sein und intuitiv klar sein.
    
    Damit diese Eigenschaft umgesetzt wird, könnte man die zukunftigen Anwendungen festlegen
    und daraus eine gute selbsterklärende Schnittstelle erstellen. 

    Jeder Ablauf eines Systemtests kann mit folgender Schema beschrieben werden:
    \begin{itemize}
        \item 1. Erstellen des Servers mit gewünschten Netzwerkeinstellungen
        \item 2. Parametrieren/Festlegen des gewünschten Verhaltens des Servers
        \item 3. Server starten
        \item 4. Festlegen die Bedingungen für den erfolgreichen Test 
        \item 5. Warten bis der Test abgeschlossen wird
        \item 6. Ergebnisse validieren
        \item 7. Alle Instanzen löschen
    \end{itemize}

