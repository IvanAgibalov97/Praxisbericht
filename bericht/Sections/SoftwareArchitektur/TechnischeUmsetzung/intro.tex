
    Der OCPP Backend Server wurde anhand ``Clean Architecture'' projektiert und entsprechend umgesetzt. 
    Im Kapitel werden die verwendeten Schichten beschrieben, welche Eigenschaften sie besitzen und wie sie miteinander verbunden sind.
    Auch wird kurz gezeigt wie die Testbarkeit auf allen Niveaus (Unit, Integration und Systemtests), 
    Änderbarkeit und Erweitbarkeit der Software ist.    
    \import{./images/}{circle_1}
    \footnotetext{Eigene Quelle}
    Beschreibung der Darstellung:
    Jede Komponente bringt in das gesamt Programm folgende Teile:
    \begin{itemize}
        \item Port - z.B. WebSocket Server aufzubauen
        \item Adapter  - umwandeln der ankommenden Nachrichten bzw. Ereignisse in die Typen definierten im Domain
        \item Controller - definiert alle Tätigkeiten, die die Komponente machen könnte
        \item UseCase - definiert den Ablauf an Tätigkeiten (Interactoren) beim Geschehen eines Ereignisses
        \item Interactors - Hülle für alle definierten Tätigkeiten im Controller
        \item Domain - definiert Typen benutzten der Komponente und deren Basic Verhalten, auch die Ereignisse die vom Dispatcher verteilt werden
    \end{itemize}
