Im Kapitel werden Teile der Anwendung beschrieben, welche Eigenschaften sie besitzen und wie sie miteinander verbunden sind.
Auch wird kurz gezeigt wie die Testbarkeit (Unit-, Integration- und Systemtests), 
Änderbarkeit und Erweiterbarkeit der Software ist.    
Ein wichtiger Teil der Beschreibung ist der Datenfluss im Programm.

\import{./images/}{circle_1}
Beschreibung der Darstellung:

Jede Komponente bringt in das gesamt Programm folgende Teile:
\begin{itemize}
    \item Port - hat die Aufgabe die Schnittstelle nach Außen aufzubauen und die Verbindungen zu kontrollieren (z.B. WebSocket Server, Datenbank).
    \item Adapter  - hat die Aufgabe ankommenden Ereignisse vom Port an den dazugehörigen Controller und in die andere Richtung zu übersetzen (z.B. Presenter im \textbf{MVC} \ref{kap:MVC})
    \item Controller - besitzet alle Informationen, die den Zustand jeweiliger Komponente (\textbf{Controller} + \textbf{Adapter} + \textbf{Port}) abbilden.
    Controller ermöglicht auch die dazugehörige Schnittstelle zu verwenden (z.B. eine Nachricht abschicken).
    \item UseCase - beschreibt den Vorgang beim Auslösen eines Erreignisses, welches sie abonniert haben. Die
    vordefinierten UseCases dürfen nur \textbf{Interactors} verwenden, um andere \textbf{Controller} anzusprechen.
    \item Interactors - Eine atomare Operation im Programm (die Operation lässt sich nicht mehr sinnvoll im Rahmen
    der Anwendung aufteilen). Für jede Methode des \textbf{Controllers}, die vom \textbf{UseCase} benutzt wird, gibt es ein \textbf{Interactor}
\end{itemize}
