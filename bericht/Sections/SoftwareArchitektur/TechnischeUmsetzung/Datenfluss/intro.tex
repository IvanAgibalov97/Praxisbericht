Im System gibt es drei wichtige Datenflüsse, die durch Kombination miteinander die komplexen Abläufe im System umsetzen.
\begin{enumerate}
    \item (grün) Das Programm wird von einem externen System angesprochen (z.B. Ladesäule schickt eine OCPP Nachricht an den Server).
    \item (blau) Controller löst ein Ereignis im Dispatcher aus.
    \item (rot) Controller spricht sein Port an(z.B. Speichern der Daten in der Datenbank oder OCPP Antwort abschicken). Dieser Datenfluss ist der umgekehrte Datenfluss ``1''.
\end{enumerate}
Die einfache Kombination aus drei Datenflüssen sieht so aus:
\import{./images/}{circle_2}

Jeder Punkt in der Darstellung repräsentiert eine Stelle, in der die Daten bearbeitet und weitergegeben werden.
