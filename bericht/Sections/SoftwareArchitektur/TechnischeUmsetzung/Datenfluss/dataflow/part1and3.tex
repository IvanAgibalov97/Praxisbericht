Darstellung des Datenflusses \textbf{1} als \textbf{sequencediagram}:

\begin{figure}[H]
    \begin{sequencediagram}
        \newthread{A}{Externe}
        \newthread{B}{Port (A)}
        \newinst[1]{C}{Adapter (B)}
        \newinst[1]{D}{Controller (C)}

        \begin{messcall}{A}{Message}{B}
            \begin{messcall}{B}{Message}{C}
                \begin{messcall}{C}{Message}{D}
                    
                \end{messcall}
            \end{messcall}
        \end{messcall}
    \end{sequencediagram}
    \caption{Sequencediagramm vom Datenfluss \textbf{1} Grün}
\end{figure}

Darstellung des Datenflusses \textbf{3} als \textbf{sequencediagram}:
\begin{figure}[h]
    \begin{sequencediagram}
        \newthread{A}{Controller (K)}
        \newinst[1]{B}{Adapter (L)}
        \newinst[1]{C}{Port (M)}
        \newthread{D}{Externe}
        \begin{messcall}{A}{Message}{B}
            \begin{messcall}{B}{Message}{C}
                \begin{messcall}{C}{Message}{D}
                    
                \end{messcall}
            \end{messcall}
        \end{messcall}
    \end{sequencediagram}
    \caption{Sequencediagramm vom Datenfluss \textbf{3} Rot}
\end{figure}\\
\newpage
\noindent Darstellung der Datentransformation:

\noindent \textbf{Controller} - \textbf{Adapter}: Alle Informationen werden mittels Struktruren übergeben, die in jeweiliger Anwendung definiert sind.
D.h. für verschiedene Anwendung, die unterschiedliche Struktruren definieren, wird das übergebene Objekt anderes aussehen.

\noindent Ein Beispiel der übergebenen Information an den \textbf{Controller} von \textbf{Adapter} 
und an den \textbf{Adapter} vom \textbf{Controller} für OCPP Kommunikation kann wie folgt aussehen:
\begin{lstlisting}[language=json,firstnumber=1]
    OCPP20Message({{
        name : "BootNotification",
        type : "Response",
        payload : BootNotification({
            currentTime : Date(Thu Jul 28 2022 14:26:49 GMT+0200),
            interval : 30,
            status : "Rejected"    
        })
    })
\end{lstlisting}

\noindent \textbf{Adapter} - \textbf{Port}: Alle Informationen, 
die gesendet werden (in dem Fall ``message''), werden in der verständlichen Form (sie muss nicht mehr in der Anwendung geändert werden)
für den Port an Port weitergegeben.
Falls es notwendig ist, müssen alle Informationen über den Empfänger oder Sender der Nachricht weitergegeben werden, 
sodass der Port die entsprechende Verbindung zuordnen kann. 
Die Struktur der übergebenen Information wird durch den Port und das benutzte Übertragungsprotokoll bestimmt. 

\noindent Ein Beispiel der übergebenen Information an den \textbf{Adapter} vom \textbf{Port} 
und an den \textbf{Port} vom \textbf{Adapter} im Falle einer OCPP Nachricht:
\begin{lstlisting}[language=json,firstnumber=1]
{
    message : "[3, 'message_id_of_request', {currentTime : 'Thu Jul 28 2022 14:26:49Z', interval : 30, status : 'Rejected'}]"
}
\end{lstlisting}