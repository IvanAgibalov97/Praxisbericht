Darstellung des Datenflusses \textbf{2} als \textbf{Sequencediagram}:

\begin{figure}[h]
    \begin{sequencediagram}
        \newthread{A}{Controller 1 (C)}
        \newinst[1]{B}{Dispatcher (D)}
        \newinst[1]{C}{UseCase (E)}
        \newinst[1]{D}{Interactor I (G)}
        \newinst[1]{E}{Controller J (C)}
        
        \begin{messcall}{C}{subscribe ``Event''}{B}
        \end{messcall}

        \begin{messcall}{A}{Event}{B}{}
                \begin{messcall}{B}{Event}{C}{}
                    \begin{sdblock}{Loop}{Until sequence of use case done}
                        \begin{call}{C}{Handle I}{D}{result}
                            \begin{call}{D}{Handle I}{E}{result}
                            \end{call}
                        \end{call}
                    \end{sdblock}
                \end{messcall}
        \end{messcall}
      \end{sequencediagram}
      \caption{Sequencediagramm vom einfachen Datenfluss \textbf{2} Blau}
      \label{fig:seqDiagBlue}
\end{figure}

Wenn ein Ereignis den \textbf{Controller} erreicht, wird \textbf{Dispatcher} darüber informiert. 
Ein oder mehrere \textbf{UseCases} haben bereits dieses Ereignis beim \textbf{Dispatcher} abonniert.
\textbf{Dispatcher} informiert alle auf das Ereignis abonierte \textbf{UseCases}. 
Jeder \textbf{UseCase} kann seinen eigenen Verhalten auf das Event definieren unabhängig voneinander.
\textbf{UseCase} definiert einen Ablauf an \textbf{Interactoren},
die wie vorgeschrieben ausgeführt werden. Jeder \textbf{Interactor} ruft eine Methode von einem \textbf{Controller} auf 
und das Ergebnis wird an \textbf{UseCase} zurückgegeben.

Dabei es gibt zwei Möglichkeiten wie das Ereignis vom \textbf{Controller J} die \textbf{Interactor I} erreichen kann:
\begin{enumerate}
    \item synchron - der Rückgabewert ist das Ergebnis der aufgerufenen Methode (siehe Darstellung \ref{fig:seqDiagBlue})
    \item asynchron - auf die dazugehörige Antwort vom Port wird gewartet(z.B. auf OCPP Response warten, wenn man ein OCPP Request abschickt)
\end{enumerate}
\newpage
Darstellung der 2. Möglichkeit:
\begin{figure}[h]
    \begin{sequencediagram}
        \newthread{U}{UseCase (E)}
        \newinst[1]{A}{Interactor I (G)}
        \newinst[2]{C}{Dispatcher (D)}
        \newinst[1]{B}{Controller J (C)}
        \newthread{D}{Externe}
        
        \begin{call}{U}{Message}{A}{Response}
        
        \begin{messcall}{A}{Message}{B}{}
            \begin{messcall}{A}{subscribe 'Response'}{C}{}
                
            \end{messcall}
            \begin{messcall}{B}{Message}{D}{}
            \end{messcall}
        \end{messcall}
        \begin{messcall}{D}{Response}{B}{}
            \begin{messcall}{B}{Response}{C}{}
                \begin{messcall}{C}{Response}{A}{}

                \end{messcall}
            \end{messcall}

            \begin{messcall}{A}{unsubscribe 'Response'}{C}{}
            \end{messcall}
        \end{messcall}
        
            
        \end{call}
    \end{sequencediagram}
    \caption{Sequencediagramm vom komplexen Datenfluss \textbf{2} Blau}
    \label{fig:dataFlowKomplexInteractor}
\end{figure}

Der Rückgabewert wird beim synchronen Funktionsaufruf zurückgegeben, wie in der Abbildung ~\ref{fig:seqDiagBlue} dargestellt.
Der Rückgabewert beim asynchronen Funktionsaufruf, wird wie folgt definiert:
Der Aufgerufene \textbf{Interactor} ruft eine Methode vom \textbf{Controller} auf, der die Nachricht an den externen Teilnehmer abschickt. 
Gleich danach abonniert der \textbf{Interactor} die Antwort auf die abgeschickte Nachricht. Wenn die Antwort ankommt, landet sie beim \textbf{Dispatcher},
die alle Abonnierten darüber informiert, unteranderem auch den \textbf{Interactor}. 
\textbf{Der Interactor} gibt diese Antwort als Rückgabewert der Funktionsaufruf