Darstellung des Datenflusses \textbf{3} als \textbf{sequencediagram}:
\begin{figure}[h]
    \begin{sequencediagram}
        \newthread{A}{Controller (K)}
        \newinst[1]{B}{Adapter (L)}
        \newinst[1]{C}{Port (M)}
        \newinst[3]{D}{Externe}
        \begin{messcall}{A}{Message}{B}
            \begin{messcall}{B}{Message}{C}
                \begin{messcall}{C}{Message}{D}
                    
                \end{messcall}
            \end{messcall}
        \end{messcall}
    \end{sequencediagram}
\end{figure}\\
Darstellung der Datentransformation:

\noindent Controller - Adapter: Alle Informationen werden mittels Struktruren übergeben, die in jeweiliger Anwendung definiert sind.
D.h. für verschiedene Anwendung, die unterschiedliche Struktruren definieren, wird das übergebene Objekt anderes aussehen.

\noindent  Ein Beispiel der übergebenen Information sieht wie folgt aus:
\begin{lstlisting}[language=json,firstnumber=1]
    OCPP20Message({
        destination: {
            charger : Charger({id : "some_unique_charger_id"})
        },
        message : Message({
            name : "BootNotification",
            type : "Response",
            payload : BootNotification({
                currentTime : Date(Thu Jul 28 2022 14:26:49 GMT+0200),
                interval : 30,
                status : "Rejected"    
            })
        })
    })
\end{lstlisting}

\noindent Adapter - Port: Alle Informationen, die gesendet werden (in dem Fall ``message''), werden in der verstandlichen Form (sie muss nicht mehr in der Anwendung geändert werden)
für den Port an Port weitergegeben.
Falls es notwendig ist, müssen müssen alle Informationen über das Ziel der Nachricht weitergegeben werden, 
sodass der Port die entsprechende Verbindung zuordnen kann. 
Die Struktur der übergebenen Information wird durch den Port und das benutzte Übertragungsprotokoll bestimmt. 

\noindent Ein Beispiel der übergebenen Information im Falle einer OCPP Nachricht:
\begin{lstlisting}[language=json,firstnumber=1]
{
    destination : {
        chargerId : "some_unique_charger_id"
    }
    message : "[3, 'message_id_of_request', {currentTime : 'Thu Jul 28 2022 14:26:49Z', interval : 30, status : 'Rejected'}]"
}
\end{lstlisting}