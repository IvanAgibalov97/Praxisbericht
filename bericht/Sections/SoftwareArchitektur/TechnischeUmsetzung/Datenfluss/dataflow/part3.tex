Darstellung des Datenflusses \textbf{3} als \textbf{sequencediagram}:
\begin{figure}[h]
    \begin{sequencediagram}
        \newthread{A}{Controller}
        \newinst[1]{B}{Adapter}
        \newinst[1]{C}{Port}
        \newinst[3]{D}{Externe}
        \begin{call}{A}{Message}{B}{return result}
            \begin{call}{B}{Message}{C}{return result}
                \begin{messcall}{C}{Message}{D}{}
                    
                \end{messcall}
            \end{call}
        \end{call}
    \end{sequencediagram}
\end{figure}\\
Darstellung der Datentransformation:\\
Controller - Adapter: Alle Informationen werden als Objekte übergeben, die im Domain definiert werden müssen.
\begin{lstlisting}[language=json,firstnumber=1]
    OCPP20Message({
        destination: {
            chargerId : "some_unique_charger_id"
        },
        message : {
            name : "BootNotification",
            type : "Response",
            payload : BootNotification({
                currentTime : Date(Thu Jul 28 2022 14:26:49 GMT+0200),
                interval : 30,
                status : "Rejected"    
            })
        }
    })
    \end{lstlisting}
    Adapter - Port: Alle Informationen, die gesendet werden (in dem Fall ``message''), werden in der verstandlichen Form (sie muss nicht mehr geändert werden)
    für den Port an Port weitergegeben.
    Über das Ziel müssen alle Informationen weitergegeben werden, so dass Port die entsprechende Verbindung zuordnen kann. 

    \begin{lstlisting}[language=json,firstnumber=1]
    {
        destination : {
            chargerId : "some_unique_charger_id"
        }
        message : "[3, 'message_id_of_request', {currentTime : 'Thu Jul 28 2022 14:26:49Z', interval : 30, status : 'Rejected'}]"
    }
\end{lstlisting}