Ein wichtiger Bestandteil jeder Software ist das Logging von unterschiedlichen Ereignissen in der Software.
Das Ziel vom Logging ist später von den Softwareentwicklern verschiede Fehler in der Software so schnell wie möglich zu finden und
zu beseitigen.

Dafür muss es möglich sein mit den Logs die Fehler so genau wie möglich in der Software zu lokalisieren 
(welche Komponente oder welche Methode den Fehler hervorgerufen hat) 
und welche Ereigniskette den jeweiligen Fehler hervorgerufen hat.

In dem Kapitel \ref{kap:Dataflow} sind alle möglichen Wege basierend auf der beschriebenen Struktur 
im Kapitel \ref{kap:Structur} für die ankommenden Ereignissen in der Software beschrieben.

Die untere Abbildung \ref{fig:FullDataFlow} zeigt den kompleten Ablauf beim Geschehen eines Ereignisses.
Eine mögliche Systematisierung des Loggierens in der Appliaction wäre: 
\begin{itemize}
    \item man zeichnet in jeder Komponente den Inhalt des ankommenden Ereignisses auf
    \item man zeichnet in jeder Komponente den Inhalt des ausgehenden Ereignisses auf
    \item man zeichnet alle Komponenten auf, an die das Ereignis weitergegeben wird
\end{itemize}

Somit lässen sich die Fehler auf die Komponente genau lokalisieren, d.h. man kann einen entsprechenden Unittest schreiben, 
der diesen Fehler abdeckt, bzw. einen Integrationtests, da man auch den Ablauf des Ereignisses kennt.

\begin{figure}[H]
    \centering
    \includegraphics[width=12cm]{./images/FullDataFlow.png}
     \caption[Kompletter Datenfluss]{Kompletter Datenfluss \footnotemark}
     \label{fig:FullDataFlow}
\end{figure}
\footnotetext{Eigene Quelle}
