
    Bei diesem Szenario wird betrachtet, dass ein Ereignis, für das neues Verhalten geändert werden soll, an \textbf{Dispatcher} ankommt und entspechend an alle dafür 
    verantwortlichen \textbf{UseCases} weiterleitet. 
    Es gibt hier zwei Möglichkeiten:
    \begin{enumerate}
        \item bestehenden \textbf{UseCase} um die neue Funktionalität erweitern.
        \item neuen \textbf{UseCase}  erstellen, das das gleiche Ereignis handelt.
    \end{enumerate}

    Bei der ersten Möglichkeit ist das komplette Verhalten für ein Ereignis an einem Ort definiert und entspechend mit Unittests abdecken lässt.

    Die zweite Möglichkeit streut das Verhalten für ein Ereignis im Projekt. Das verbessert die Lesbarkeit des jeweiligen Teils,
    jedoch die Schwierigkeit bringt alle solche Teile im Projekt zu finden. Das Gesamtverhalten lässt sich erst mittels
    einem Integrationtest abdecken, was eine mögliche Fehlersuche erschweren kann. 
    Für den Fall, dass das neue Verhalten unabhängig von dem bestehenden Verhalten ablaufen soll ist es eine gute Möglichkeit.