
    Die Architektur lässt sich in zwei wesentlichen Teilen zerlegen
    \begin{itemize}
        \item Anbindung an Infrastruktur um das Programm (Port - Adapter - Controller)
        \item Innere Logik des Programms (Controller - Dispatcher - UseCase - Interactor)
    \end{itemize}

    Beispiele für die Infrastruktur sind: Datenbank, Persistenz, Schnittstellen (HTTP, USB usw)

    Bei solcher Aufteilung ergeben sich folgende Vorteile:
    \begin{itemize}
        \item Innere Logik des Programms lässt sich mittels Integrationstests unabhängig von Schnittstellen abdecken.
        Damit ist die Laufzeit von jedem einzelnen Test ohne reelen Schnittstellen ist schneller als mit reelen Schnittstellen und 
        man hat das gleiche Ergebnis bezüglich des Verhaltens des Programms.
        \item Die Innere Logik ist nicht an Schnittstellen gebunden, 
        somit können alle Schnittstellen mit wenig Aufwand getauscht werden.
    \end{itemize}
