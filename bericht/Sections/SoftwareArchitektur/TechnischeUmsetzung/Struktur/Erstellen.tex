In den Kapiteln \ref{Controller-Dispatcher-UseCase-Interactor} und \ref{Port-Adapter-Controller} 
werden die fertigen Strukturen beschrieben, diese Strukturen müssen am Anfang des Programms erstellt
und miteinander verbunden werden. Für das Erstellen wird Pattern \textbf{Builder} verwendet (siehe Kapitel \ref{kap:gof:builder}).

Das Erstellen von der Struktur findet im Hauptprogramm statt und lässt sich in drei Schritte aufteilen:
\begin{itemize}
    \item Erstellen aller Instanzen
    \item Verknüpfen aller Instanzen miteinander
    \item Starten aller Instanzen
\end{itemize}

Das Erstellen aller Instanzen lässt sich in zwei weitere Schritte aufteilen, die bedingt voneinander abhängen.
\begin{itemize}
    \item Kern (Controllers + Dispatcher + UseCases + Interactors)
    \item Schnittstellen (Port + Adapter + Controller)
\end{itemize}

Damit auch Integrationstests für den kompletten Core und jede Schnittstelle möglich ist, ist es sinnvol, dass beide Schritte
explicit ausgeführt werden.

Ein möglicher Ablauf ist in Abbildung \ref{fig:ADCreate} dargestellt:
\begin{figure}[H]
    \centering
    \includegraphics[width=12cm]{./images/Erstellen AD.png}
     \caption[Ablaufiagramm Erstellen der Struktur]{Ablaufiagramm Erstellen der Struktur}
     \label{fig:ADCreate}
\end{figure}

Mit diesem Ablauf können mehrere Hauptprogramme erstellt werden, die verschiedene Anwendungen für verschiedene Zwecke erstellen.
Z.B. ein Framework braucht keine reelen Anknüpfungen an die Infrastruktur (z.B. Datenbank) im Vergleich zu Standalone Anwendung.
Oder es kann verschiedene Hauptprogramme für verschiedene Datenbanken geben.
