% Bevor man anfängt über die Software Architektur zu reden, muss man sie erstmal definieren.
Es existiert keine einheitliche Definition einer Software Architektur. Verschiedene Authoren definieren es auch unterschiedlich.

Robert Martin definiert es als ein Gegenstand mit bestimmten Eigenschaften zu definieren.\\
\textit{The architecture of a software system is the shape given to that system by those who build it. 
The form of that shape is in the division of that system into components, the arrangement of those components, 
and the ways in which those components communicate with each other.} \cite[136]{cleanArchitecture}

Ralph Jonson definiert Software Architektur aus Sicht eines Projektes.\\
\textit{Architecture is the set of design decisions that must be made early in a project}
\cite{MF_WhatIsSA}

Beide Definitionen schließen sich gegenseitig nicht aus und
meiner Meinung nach müssen beide gleichwertig bei der Planung einer Anwedung beachtet werden.

In dem ersten Teil des Kapitels wird die Softwarearchitektur aus Sicht eines Projektes betrachtet.
Hierbei wird kurz beschrieben, welche Auswirkungen die Menge an inverstierter Zeit in die Architektur auf ein Projekt hat.

Im zweiten Teil des Kapitels, werden folgende Themen der Architektur im allgemeinen näher beleuchtet:
\begin{itemize}
    \item Aufteilung der Anwendung in einzelne Teile
    \item Testbarkeit und Erweitbarkeit einzelner Teile des Programms
    \item Kommunikation zwischen den einzelnen Teilen
\end{itemize}