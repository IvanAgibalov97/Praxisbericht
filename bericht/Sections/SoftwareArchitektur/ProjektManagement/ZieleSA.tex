Jede Software erfüllt bestimmte Anforderungen, die von Außen gestellt werden. 
Diese Anforderungen sind meistens von nicht Softwareentwicklern definiert und beziehen sich auf nicht Informatikgebiete 
    (z.B. Banksoftware oder eine Smartphone Anwendung). 
    Das Erfüllen von diesen Anforderungen ist das Ziel von jedem Softwareprojekt.
    Jedoch bei der Umsetzung entstehen viele Herausforderungen, die für die Außenstehenden nicht bekannt oder nicht relevant sind.
    (Auswahl einer Datenbank, Optimierung der Ressourcenverwendung, innere Struktur der Anwendung usw.). 
    Die dabei getroffenen Entscheidungen haben eine große Auswirkung auf die Entwicklungszeit bzw.
    benötigten Ressourcen und somit kann das Ziel der Software Architektur wie folgt definiert werden:

    \textit{The goal of software architecture is to minimize the human resources required
    to build and maintain the required system}\cite[5]{cleanArchitecture}

    Diese Aussage lässt sich sehr einfach überprüfen, indem festgestellt wird, 
    ob jede neue Anforderungen an der Software mehr Ressourcen verbraucht als die vorherigen.

    \textit{The strategy [...] is to leave as many options open as possible, for as long as possible}
    \cite[136]{cleanArchitecture}

    Beispiele für solche Entscheidungen sind:
    \begin{itemize}
        \item Datenbanksystem
        \item Transferprotokoll zu der Benutzeroberfläche (z.B. HTTP oder WS) falls vorhanden
        \item Wie und wo die Loggingdaten gespeichert werden (in einer Datei, Datenbank oder externe Server)
    \end{itemize}

    Auch die anderen Tätigkeiten, die nicht direkt das Programmieren betreffen, sind von den Entscheidungen in der SoftwareArchitektur betroffen:
    \begin{itemize}
        \item Deployment (Aufsetzung) der Software.
        \item Maintenance (Unterstützung) der Software.
    \end{itemize}

    Deployment der Software beinhaltet die Kosten die durch das Aufsetzen der neuen Version der Software entstehen.\\
    Maintenance der Software beinhaltet die Kosten, die nach dem Beenden der Entwicklung bei kleineren Erweiterungen und Änderungen des Systems entstehen.

