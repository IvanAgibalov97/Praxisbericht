Jedes Teil der Software wird in seinem Lebenszyklus mehrmals geändert. 
Um die Funktionalität der neuen Version zu verifizieren, muss sie getestet werden.
Es ist von Interesse diese Aufgabe zu automatisieren. 
Wie in den früheren Kapitels bereits beschrieben wurde, am schnellsten findet man die Bugs, falls vorhanden, mit Unittests. 
Bei den Unittests müssen die Module (z.B. einzelne Klassen in Falle von OOP Sprachen) in verschiedenen Umgebungen überprüft werden. 
Das heißt, dass die Zustände von benutzten Modulen müssen einfach zu simulieren sein.
Dies erfordert eine Planung der Softwarearchitektur im Voraus, 
um diese Eigenschaft zu implementieren um im Laufe der Entwicklung, Zeit durch automatisierte Tests zu sparen.
