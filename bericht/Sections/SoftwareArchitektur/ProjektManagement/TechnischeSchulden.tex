Bei den Änderungen oder Erweiterungen eines Systems entsteht oft ein Overhead, welcher durch die ``Unsauberkeit'' des bestehenden Programms verusacht wird.

Dieser Overhead wird als technische Schulden (en. : Technical Debts) bezeichnet.

Die technischen Schulden entstehen dadurch, dass bei der Entwicklung eines Teiles des Systems wurde 
von den Entwicklungsteam weniger Zeit für nicht gewinnbringende Aufgaben inverstiert wurde.
Beispiele für solche Tätigkeiten sind:
\begin{itemize}
    \item Unittests
    \item Dokumentieren 
    \item Code Review
\end{itemize}

Beispiele für Technische Schulden sind:
\begin{itemize}
    \item Alte Funktionalitäten funktionieren nach der Änderung nicht mehr
    \item Aufdeckung eines Bugs erst nach einer gewissen Zeit in Produktionsversion der Software
    \item Implementieren der neuen Funktionalitäten verbraucht deutlich mehr Zeit
\end{itemize}

Eine klare Struktur der Software reduziert die Menge an technischen Schulden, 
die die Weiterentwicklung in der Zukunft verlangsamen. 

Die Softwareentwickler können die ankommenden Aufgaben erledigen
\begin{itemize}
    \item man hat bereits Vorgaben wie die Kommunikationswege zwischen den Modulen ist
    \item wie die Module benannt werden sollen
    \item an welchen Stellen das Modul in das System hinzugefügt werden soll
    \item die Menge an durch den "Zufall" entstehenden Bugs in anderen Teilen des Programms ist minimal
\end{itemize}

Durch die bereits definierten Kommunikationswege zwischen den Modulen, muss weniger dokumentiert werden.
Mit weniger Dokumentation können die gesuchten Informationen schneller gefunden werden.

Durch die einheitliche Bezeichnung der Teile des Modules ist es möglich aus dem Namen des Modules seine Aufgaben ableiten.

Daher ist es vom Vorteil vor Begin der Umsetzung des Softwaresystems, die oben gennanten Aufgaben zu lösen,
denn mit zunehmender Lebenszeit der Software nimmt die Änderungszeit zu.

Somit lassen sich die vorhandenen Ressourcen effizienter eingesetzen.