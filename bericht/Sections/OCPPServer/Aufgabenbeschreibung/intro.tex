Als Beispiel der Anwendung der beschriebenen Architektur dient OCPP 1.6. Server.
Es sollen insgesamt mehrere unterschiedliche Anwednungen entstehen, die für unterschiedliche 
Zwecke gedacht sind.

Das sind:
\begin{itemize}
    \item Testframework für die automatisierten Systemtests.
    \item Eigenständiger OCPP1.6 und später OCPP2.0 Server (mit und ohne Datenbank).
    \item Bibliothek, die in einer anderen Anwendung benutzt wird.
\end{itemize}

Der Server soll mehrere Bereiche unterstützen, 
die in Abhängigkeit von jeweiligem Zustand des Bereiches das Verhalten des Servers ändern.

Die gewünschte Bereiche, die der Server unterstützen soll, sind:
\begin{itemize}
    \item Ladesäuleverwaltung: die angefragten Verbindungen von bekannten Ladesäulen akzeptieren bzw. von unbekannten Ladesäulen ablehnen.
    \item Benutzerverwaltung: die Authentifizierung von bekannten Benutzern an Ladesäulen zulassen bzw. von unbekannten Benutzern ablehnen.
    \item Preisverwaltung: unterschiedliche Preise für unterschiedliche Benutzer emöglichen.
    \item Ladevorgangverwaltung: Ladevorgänge serverseitig unterstützen.
    \item Benutzeroberflächeverwaltung: ermöglichen alle Bereiche des Servers manuell ändern zu können.
    \item Ladesäulekommunikationverwaltung: die Kommunikation mit den Ladesäulen mittels OCPP Protokoll ermöglichen.
    \item Loggingverwaltung: die wichtigen Ereignisse im Server aufzeichnen.
\end{itemize}