Zu Begin der Praxisphase war die Entwicklung in der Abteilung, in der ich meine Praxisphase gemacht habe, und Testen an vielen Stellen an eine externe Schnittstelle gebunden.
Um diese Abhängigkeit zu reduzieren und die Möglichkeit zu bekommen eigene Funktionalitäten hinzuzufügen, bekamm ich die Aufgabe
einen OCPP 1.6. Server zu implementieren.

Zu dem Zeitpunkt gab es bereits drei Stellen, bei denen der OCPP1.6 Server Funktionalitäten in unterschiedlichem Umfang benutzt wurden.

Das sind:
\begin{itemize}
    \item Testframework für die automatisierten Systemtests von der Ladesäule
    \item Eigenständiger OCPP1.6 und später OCPP2.0 Server für die manuellen Tests. (mit einer reelen und einer gemockten Datenbank)
    \item Automatisierungstool für den ERK (Eichrechtkonformität) Prozess
\end{itemize}

Die dritte Aufgabe (Automatisierungstool für den ERK Prozess) hatte großere Priorität und somit wurde diese als erstes erledigt.
Die Aufgabe hat dabei den kleinsten Anteil an OCPP Server Funktionalitäten gebraucht.

Anschließend wurden die Teilaufgaben erledigt, aufgrund der großen Überschneidung der programmiertechnischen Anforderungen, 
konnten Teile der vorhergehenden Aufgabe übernommen werden.