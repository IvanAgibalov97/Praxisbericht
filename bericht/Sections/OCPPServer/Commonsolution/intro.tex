\import{./images/solutions}{common}

In der Abbildung \ref{fig:commonSolution} ist die allgemeine Lösung aller im Kapitel \ref{kap:taskDescription} beschriebenen Aufgaben dargestellt.
Die Darstellung ist nach den Regeln vom Kapitel \ref{kap:commonArchitectureDescription} gebaut.

Die allgemeine Lösung aller Aufgaben besitzt vier Schnittstellen (Ports):
\begin{itemize}
    \item Datenbank (um verschiedene Daten zu speichern)
    \item HTTP (für Benutzerinteractionen)
    \item Filesystem (für Logging)
    \item Websocket (für OCPP Schnittstelle)
\end{itemize}

Die 4 Schnittstellen (Ports) werden von insgesamt 8 Controllers benutzt.
Die Datenbank wird von Charger-, User-, Transaction-, Price- und DB-Controller benutzt.
Die HTTP Schnittstelle, in dem Fall als Server, wird nur von HTTP-Controller benutzt.
Das Filesystem wird vom Logging-Controller benutzt.
Die Websocketschnittstelle, in dem Fall als Server, wird von OCPP-Controller benutzt.

Jedes Ereignis in jedem \textbf{Controller} wird an Dispatcher weitergegeben, der alle abonnierten \textbf{UseCases} darüber informiert.

Insgesamt gibt es 32 \textbf{UseCases}. 
Davon 14 reagieren auf die Ereignisse vom ``OCPPComm'' Controller und 18 reagieren auf die Ereignisse vom ``HTPPComm'' Controller. 
Alle anderen \textbf{Controller} definieren keine eigene \textbf{UseCases} auf verschiedene Ereignisse.

\textbf{Super-Controller} besitzt die Aufgabe, den Zugriff auf die anderen \textbf{Contollers} zu ermöglichen. 
In der Lösung wird das nur von \textbf{UI-Controller} verwendet, somit besteht die Möglichkeit das Programm über \textbf{HTPP-Schnittstelle} zu verwalten. 
(z.B. Zustände von den \textbf{Controllern} abzufragen).

\textbf{OCPP-Controller}

\textbf{UI-Controller}

\textbf{Logger-Controller}

\textbf{DB-Controller}

\textbf{Chager-Controller}

\textbf{User-Controller}

\textbf{Transaction-Controller}

\textbf{Price-Controller}