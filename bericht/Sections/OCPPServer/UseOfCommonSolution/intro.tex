Jede umgesetzte Lösung benutzt das im Kapitel \ref{kap:commonSolution} beschriebene Verhalten und Struktur. 

Der Unterschiede zwischen den einzelnen Lösungen sind:
\begin{itemize}
    \item \textbf{Ports}, die von \textbf{Controllers} benutzt werden
    \item Fassade der Anwendung (betrifft nur Library und Framework)
\end{itemize}

Jeder \textbf{Controller} außer ``Super''-Controller kann eine reele Schnittstelle benutzen, 
falls sie in der Anwendung gebraucht wird. In jeder Anwendung kann eine Schnittstelle von mehreren \textbf{Controllers} verwendet werden.
Das lässt sich zum Beispiel mittel ISP (siehe Kapitel \ref{kap:ISP}) umsetzen.

Jede Schnittstelle kann drei Versionen haben:
\begin{itemize}
    \item reele Schnittstelle
    \item Schnittstelle ist gefälscht. Zum Beispiel statt Datenbank Cache verwenden. Das Verhalten der gefälschten Schnittstelle ist wichtig.
    \item Schnittstelle wird nicht benutzt, sie muss aber trotzdem übergeben werden, 
    damit Parameterliste erfüllt wird. Das Verhalten der Schnittstelle ist unwichtig.
\end{itemize}

\newpage
Jede Anwendung soll OCPP Kommunikation serverseitig können. 
Aus diesem Grund wird OCPP Schnittstelle (in dem Fall Websocket-Server) immer reel und gleich sein.

Zusammengefasst sieht das gleiche Teil jeder Anwendung so aus:
\import{./images/solutions}{CommonPartOfEverySolution}