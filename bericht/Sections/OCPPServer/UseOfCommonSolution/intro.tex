Jede umgesetzte Lösung benutzt das im Kapitel \ref{kap:commonSolution} beschriebene Verhalten und Struktur. 

Der Unterschiede zwischen den einzelnen Lösungen sind:
\begin{itemize}
    \item \textbf{Ports}, die von \textbf{Controllers} benutzt werden
    \item Fassade der Anwendung (betrifft nur Library und Framework)
\end{itemize}

Jeder \textbf{Controller} außer ``Super''-Controller kann eine reele Schnittstelle benutzen oder die Schnittstelle kann weggelassen werden, 
falls es in der Anwedung nicht verwendet wird. In jeder Anwendung kann eine Schnittstelle von mehreren \textbf{Controllers} verwendet werden.
Das lässt sich zum Beispiel mittel ISP (siehe Kapitel \ref{kap:ISP}) umsetzen.

Jede Anwendung soll OCPP Kommunikation serverseitig können. 
Aus diesem Grund wird OCPP Schnittstelle (in dem Fall Websocket Schnittstelle) immer reel und gleich sein.

Zusammengefasst sieht das gleiche Teil jeder Anwendung so aus:
\import{./images/solutions}{CommonPartOfEverySolution}

Die allgemeine Lösung aller Aufgaben besitzt vier Schnittstellen (Ports):
\begin{itemize}
    \item Datenbank (um verschiedene Daten zu speichern)
    \item HTTP (für Benutzerinteractionen)
    \item Filesystem (für Logging)
    \item Websocket (für OCPP Schnittstelle)
\end{itemize}

Die 4 Schnittstellen (Ports) werden von insgesamt 8 Controllers benutzt.
Die Datenbank wird von Charger-, User-, Transaction-, Price- und DB-Controller benutzt.
Die HTTP Schnittstelle, in dem Fall als Server, wird nur von HTTP-Controller benutzt.
Das Filesystem wird vom Logging-Controller benutzt.
Die Websocketschnittstelle, in dem Fall als Server, wird von OCPP-Controller benutzt.