\section{Aufgabenstellung}

Hier steht iwas zur Aufgabestellung

Bedarf:
\begin{itemize}
    \item Testframework für die Integrationstest von der Ladesäule
    \item Eigenständiges OCPP1.6 und später OCPP2.0 Server für die manuellen Tests
    \item ERK (Eichrechtkonformität) automatisiert überprüfen
\end{itemize}

Lösung:
Beide Anforderungen sind sehr ähnlich zueinander =\> soweit es geht gleiches Code benutzen, nur die unterschiedlichen Teile jeweils für den Bedarf schreiben.

\subsection{Anforderungen an Standalone Server}

    Der Server wird großtenteils für die manuellen Tests benutzt. 

    \begin{itemize}
        \item Der Zustand des Servers soll für den Nutzer erreichbar sein (z.B. eine REST Schnittstelle)
        \item Der Nutzer soll in der Lage sein den Server zu parametrieren
    \end{itemize}

\subsection{Anforderungen an Testframework }
    \begin{itemize}
        \item Der OCPP Server soll den Port selber auswählen können und dann ihn zurückgeben => um mehrere Tests parallel starten zu können.
        \item Das Verhalten von dem Testserver soll geändert werden können (auch während der Tests)
        \item Das Defaultverhalten von dem Testserver soll parametrierbar sein (z.b. einen Benutzer hinzufügen)
        \item Alle Events, die den Zustand der getesteten Ladesäule aufdecken, sollen beobachtbar sein
    \end{itemize}

\subsection{Anforderungen an ERK Teil}
    \textbf{BESCHREIBUNG DER LADESAULE}

    Die Zertifizierung nach dem deutschen Eichrecht entsprechend, muss jede Ladesäule auf Eichrechtskonformität überprüft werden.
    Dieser Prozess wird immer wieder auf die gleiche Art und Weise im gleichen Umfang durchgeführt. Er beinhaltet somit ein entsprechendes Automatisierungspotential.
  
    Dafür muss für jede Ladesäule ein Ladevorgang gestartet werden (Transaction), währendderen Strom fließt und gemessen wird.   Um die Datenintegrität der Messwerte und deren Transport sicherzustellen wird der OCPP Server verwendet. Mit dessen Hilfe kann nachgewiesen werden, 
    dass die Daten nirgendwo in der Software geändert und ebenso unverändert an den Server übertragen und dort abgerechnet wurden. 
    Nach Beenden der Transaction werden mittels einer Drittsoftware die gemessenen Daten mit den Transactiondaten verglichen.

