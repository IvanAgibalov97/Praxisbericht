\section{Aufgabenstellung}

Hier steht iwas zur Aufgabestellung

Bedarf:
\begin{itemize}
    \item Testframework für die Integrationstest von der Ladesäule
    \item Eigenständiges OCPP1.6 und später OCPP2.0 Server für die manuellen Tests
    \item ERK (Eichrechtkonformität) automatisiert überprüfen
\end{itemize}

Lösung:
Beide Anforderungen sind sehr ähnlich zueinander =\> soweit es geht gleiches Code benutzen, nur die unterschiedlichen Teile jeweils für den Bedarf schreiben.

\subsection{Anforderungen an Standalone Server}

    Der Server wird großtenteils für die manuellen Tests benutzt. 

    \begin{itemize}
        \item Der Zustand des Servers soll für den Nutzer erreichbar sein (z.B. eine REST Schnittstelle)
        \item Der Nutzer soll in der Lage sein den Server zu parametrieren
    \end{itemize}

\subsection{Anforderungen an Testframework }
    \begin{itemize}
        \item Der OCPP Server soll den Port selber auswählen können und dann ihn zurückgeben => um mehrere Tests parallel starten zu können.
        \item Das Verhalten von dem Testserver soll geändert werden können (auch während der Tests)
        \item Das Defaultverhalten von dem Testserver soll parametrierbar sein (z.b. einen Benutzer hinzufügen)
        \item Alle Events, die den Zustand der getesteten Ladesäule aufdecken, sollen beobachtbar sein
    \end{itemize}

\subsection{Anforderungen an ERK Teil}
    Damit die Ladesäule ERK ist, muss dies für jeder Ladesäule überprüft werden. Dies ist immer das gleiche Prozess, das sich entsprechend automatisieren lässt.
Der Teil mit OCPP Server wird gebraucht um nachzuweisen, dass die gemessenen Daten nirgendwo in der Software geändert wurden und werden entsprechend genauso an der Server übertragen und dort abgerechnet.
Dafür muss man für jede Ladesäule einen Ladevorgang starten (Transaction), währendderen Strom fließt und gemessen wird. Die Ladesäule übertragt dabei die Sichertsschlussel an den Server um dann sicherstellen zu können,
dass die Messdaten nicht manipuliert wurden. Nach dem Beenden der Transaction werden die gemessenen Daten mit der Transactionsdaten mittels eine Drittsoftware überprüft.

