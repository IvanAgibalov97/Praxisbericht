\section{Aufgabenstellung}
Zu Begin der Praxisphase war die Entwicklung und Testen an vielen Stellen an eine externe Schnittstelle gebunden.
Um diese Abhängigkeit zu reduzieren und die Möglichkeit zu bekommen eigene Funktionalitäten hinzuzufügen, bekammm ich die Aufgabe
einen OCPP 1.6. Server zu implementieren.

Zu dem Zeitpunkt gab es bereits drei Stellen, bei denen der OCPP1.6 Server Funktionalitäten in unterschiedlichem Umfang gebraucht wurden.

Das sind:
\begin{itemize}
    \item Testframework für die automatisierten Systemtests von der Ladesäule
    \item Eigenständiger OCPP1.6 und später OCPP2.0 Server für die manuellen Tests
    \item Automatisierungstool für den ERK (Eichrechtkonformität) Prozess
\end{itemize}

Die dritte Aufgabe (Automatisierungstool für den ERK Prozess) hatte großere Priorität und somit wurde diese als erstes erledigt.
Die Aufgabe hat dabei den kleinsten Anteil an OCPP Server Funktionalitäten gebraucht.

Anschließend wurden die Teilaufgaben erledigt, aufgrund der großen Überschneidung der programmiertechnischen Anforderungen, 
konnten Teile der vorhergehenden Aufgabe übernommen werden.


\subsection{Anforderungen an den Standalone Server}
    Der Server, der im lokalen Netz auf einem Raspberry Pi läuft, soll für die manuellen Tests der Ladesäule benutzt werden.
    Dieser Server wird benutzt um die komplexeren Testfälle nachzubilden oder neue Funktionalitäten, die mit Hardware interagieren, zu testen.
    Der Server kann bei der Präsentation der Funktionalitäten genutzt werden.

    Die Anforderungen an den Server sind:
    \begin{itemize}
        \item Der Server soll eine OCPP1.6 Schnittstelle besitzen.
        \item Der Nutzer soll in der Lage sein den Server zu parametrieren (z.B. einen neuen Benutzer hinterlegen)
        \item Der Nutzer soll in der Lage sein die Nachrichten an die Ladesäule manuell verschicken zu können
    \end{itemize}

\newpage
\subsection{Anforderungen an das Testframework }
    Das Testframework soll für die automatisierten Systemtests der Software der Ladesäule (sowohl mit als auch ohne Hardware) benutzt werden.

    Die Anforderungen an das Testframework sind:
    \begin{itemize}
        \item Der OCPP Server soll den Port selber auswählen können, um mehrere Tests parallel starten zu können.
        \item Das Verhalten von dem Testserver soll geändert werden können (auch während der Tests)
        \item Das Defaultverhalten von dem Testserver soll parametrierbar sein (z.b. einen Benutzer hinzufügen)
        \item Alle Events, die den Zustand der getesteten Ladesäule aufdecken, sollen beobachtbar sein (z.B. OCPP Nachrichten, Netzwerkevents usw.)
    \end{itemize}

\subsection{Anforderungen an ERK Automatisierungstool}
    Der Zertifizierung nach dem deutschen Eichrecht entsprechend, muss jede Ladesäule auf Eichrechtskonformität (ERK) überprüft werden.
    Dieser Prozess wird immer wieder auf die gleiche Art und Weise im gleichen Umfang durchgeführt. 
    Er beinhaltet somit ein entsprechendes Automatisierungspotential.
  
    Dafür muss für jede Ladesäule ein Ladevorgang gestartet werden (Transaction), währenddessen Strom fließt und gemessen wird.   
    Um die Datenintegrität der Messwerte und deren Transport sicherzustellen wird der OCPP Server verwendet. 
    Mit dessen Hilfe kann nachgewiesen werden, 
    dass die Daten nirgendwo in der Software geändert und ebenso unverändert an den Server übertragen und dort abgerechnet wurden. 
    Nach Beenden der Transaction werden mittels einer Drittsoftware die gemessenen Daten mit den Transactiondaten verglichen.

    Die gewünschte Software soll demnächst von den Mitarbeitern im End-Of-Line benutzt werden, somit muss die Bedienbarkeit der Software sehr hoch sein,
    um die Fehlermöglichkeiten stark einzugrenzen und die Einarbeitungszeit zu reduzieren.

    Die Anforderungen an das ERK Automatisierungstool sind:
    \begin{itemize}
        \item Leichte Bedienbarkeit der Software
        \item Leicht Integrierbar in das andere Automatisierungstool
    \end{itemize}



